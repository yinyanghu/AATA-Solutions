\subsection*{Problem 31}
\noindent Reflexive: \\
Since $e \in S_n$, we know that $e \alpha e^{-1} = \alpha$ for all $\alpha \in S_n$. Thus, we have $\alpha \sim \alpha$. \\
Symmetric: \\
If there exist $\alpha \sim \beta$, it means that $\sigma \alpha \sigma^{-1} = \beta$ for some $\sigma \in S_n$. Then, we have $\alpha = \sigma^{-1} \beta \sigma = \sigma^{-1} \beta (\sigma^{-1})^{-1}$. Thus, we conclude that $\beta \sim \alpha$. \\
Transitive: \\
Suppose that we know that $\alpha \sim \beta$ and $\beta \sim \gamma$, it means that $\beta = \sigma \alpha \sigma^{-1}$ and $\gamma = \delta \beta \delta^{-1}$ for some $\sigma, \delta \in S_n$.
So $\gamma = \delta \sigma \alpha \sigma^{-1} \delta^{-1} = (\delta \sigma) \alpha (\sigma^{-1} \delta^{-1}) = (\delta \sigma) \alpha (\delta \sigma)^{-1}$ for some $\delta \sigma \in S_n$. \\

\noindent Therefore, $\sim$ is an equivalence relation on $S_n$.

